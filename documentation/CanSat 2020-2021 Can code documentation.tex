\documentclass[a4paper,10pt]{article}

\author{Joep van Dijk}
\title{CanSat 2020-2021 Can code documentation}
\date{\today}

\renewcommand{\contentsname}{Table of contents}

\newcommand{\centereditem}[1]{\item \begin{center} #1 \end{center}}

\newcommand{\isqc}{$I^2C$}


\begin{document}
	\maketitle
	
	\clearpage
	
	\tableofcontents
	
	\clearpage
	
	
	
	\section[Can]{Can overview}
		\subsection[Modules]{Modules and sensors in the Can}
			The micro controller in the Can is an Arduino Mega 2560. There are a few modules/sensors that will be connected to the Arduino. 
			\begin{itemize}
				\item The APC220 radio module, it is used to transmit and receive data. 
				\item The Adafruit BMP280 module, it measures the temperature and pressure. 
				\item The Adafruit MPU6050 module, it is an accelerometer and gyroscope, it also measures the temperature. 
				\item The NEO-6M GPS module which receives GPS data. 
				\item The HW-125 SD card module, it is used to save all the radio transmissions (and maybe some additional data). 
			\end{itemize}
			
			\vspace{1ex}
			
			\subsubsection[APC220]{The APC220 radio tranceiver module}
				The APC220 radio tranceiver module will be connected to the rx1 and tx1 serial pins. 
			
			\vspace{1ex}
			
			\subsubsection[BMP280]{The Adafruit BMP280 module}
				The BMP280 module is connected to the \isqc pins on the Arduino. The Adafruit BMP280 library\footnote{Adafruit BMP280 library: https://github.com/adafruit/Adafruit\_BMP280\_Library} uses the \isqc protocol to communicate with the BMP280 module. 
	
	
	
\end{document}